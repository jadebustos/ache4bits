\documentclass[12pt,a4paper]{report}

\usepackage{python}
\usepackage[utf8]{inputenc}
\usepackage{graphicx}
\usepackage{hyperref}

\newtheorem{example}{Example}

\begin{document}

\begin{example}
This is a quite simple example about using python within \LaTeX:
\end{example}

\begin{python}
# a ----> c
# b ----> x
#
def proporcionality(a, b, c, unit):
  sol = (c*b/a)
  # we need to print LaTeX commands to produce LaTeX output
  # utf-8 is used
  print(r'Peter is paid %d %s \ for working %d hours. He will be paid %.2f %s \ for working %d hours.\\ \\' % (b, unit, a, sol, unit, c))
  print(r'Powered by \LaTeX \ and Python.')

proporcionality(10, 120, 40, '\\texteuro')
\end{python}

\begin{example}
Matrix multiplication with Python doing the dirty work:
\end{example}

\begin{python}
import numpy as np
  
# matrix A
A = [[1,1,1],[1,2,-3],[1,-3,18]]
# matrix B
B = [[27,-21,-5],[-21,17,4],[-5,4,1]]
# lets do the multiplication
C = np.matmul(A, B)

print(r'\begin{eqnarray*}')
# print matrix A
print(r'\left( \begin{array}{ccc}')
for i in range(3):
  print(r'%d & %d & %d \\' % (A[i][0], A[i][1], A[i][2]))
print(r'\end{array} \right)')

print(r'\cdot')

# print matrix B
print(r'\left( \begin{array}{ccc}')
for i in range(3):
  print(r'%d & %d & %d \\' % (B[i][0], B[i][1], B[i][2]))
print(r'\end{array} \right)')
print(r'=')

# lets print C = A * B
print(r'\left( \begin{array}{ccc}')
for i in range(3):
  print(r'%d & %d & %d \\' % (C[i][0], C[i][1], C[i][2]))
print(r'\end{array} \right)')
  
print(r'\end{eqnarray*}')
\end{python}

\begin{example}
Using matplotlib tools. This example has been taken from \href{https://matplotlib.org/basemap/users/examples.html}{Maplotlib Documentation}:
\end{example}

\begin{python}
from mpl_toolkits.basemap import Basemap
import matplotlib.pyplot as plt
import numpy as np
# set up orthographic map projection with
# perspective of satellite looking down at 50N, 100W.
# use low resolution coastlines.
map = Basemap(projection='ortho',lat_0=45,lon_0=-100,resolution='l')
# draw coastlines, country boundaries, fill continents.
map.drawcoastlines(linewidth=0.25)
map.drawcountries(linewidth=0.25)
map.fillcontinents(color='coral',lake_color='aqua')
# draw the edge of the map projection region (the projection limb)
map.drawmapboundary(fill_color='aqua')
# draw lat/lon grid lines every 30 degrees.
map.drawmeridians(np.arange(0,360,30))
map.drawparallels(np.arange(-90,90,30))
# make up some data on a regular lat/lon grid.
nlats = 73; nlons = 145; delta = 2.*np.pi/(nlons-1)
lats = (0.5*np.pi-delta*np.indices((nlats,nlons))[0,:,:])
lons = (delta*np.indices((nlats,nlons))[1,:,:])
wave = 0.75*(np.sin(2.*lats)**8*np.cos(4.*lons))
mean = 0.5*np.cos(2.*lats)*((np.sin(2.*lats))**2 + 2.)
# compute native map projection coordinates of lat/lon grid.
x, y = map(lons*180./np.pi, lats*180./np.pi)
# contour data over the map.
cs = map.contour(x,y,wave+mean,15,linewidths=1.5)
plt.title('contour lines over filled continent background')
# create an image file with the plot
plt.savefig('contour.png')
\end{python}

% include the created image file in the document
\begin{center}
\includegraphics{contour.png}
\end{center}

\end{document}


